\chapter{Linear Programming}
Linear programming is a well studied branch of the mathematics that studies the optimization of linear functions under linear constrains. The study of linear programming started during the second part of the 1940s, as a technique military oriented problems.

We can formulate the problem in its general form as follows:

\begin{problem}
Given a cost vector $\mathbf{c}\in \Real^n$, a linear operator $\mathbf{A} \in M^{m\times n}$ 
\begin{align}
	&\min & \mathbf{c}^\top\mathbf{x} \label{eq: Linear Programming Min}&\\
	&\subject& \mathbf{A}\mathbf{x}&=\mathbf{b} & \label{eq: Linear Programming Constraints} \\
	& &\mathbf{x}&\geq 0 & \label{eq: Linear Programming cone}
\end{align} 
\end{problem}
Where $\mathbf{A}$ is a $m\times n$ matrix, and $\mathbf{b}$ is an m-dimensional column vector. The vector inequality means $\mathbf{x}\geq 0$ means that each component is nonnegative. This problem has a solution if $n>m$. 

Consider the system of equalities \eqref{eq: Linear Programming Constraints}, the vector $\mathbf{b}$ belongs to $\Real^m$.


\begin{definition}
	Given the set of $m$ simultaneous linear equations \eqref{eq: Linear Programming Constraints} with $n$ unknowns, let $\mathbf{B}$ be any nonsingular $m\times m$ submatrix made up of columns of $\mathbb{A}$. Then if all $n-m$ 
\end{definition}

\begin{definition}
	If one or more of the basic variables in a basic solution has value zero, that solution is said to be degenerate solution basic solution
\end{definition}

\begin{theorem}[Fundamental theorem of linear programming.] Given a linear program in the standard form \eqref{eq: Linear Programming Min}, \eqref{eq: Linear Programming Constraints} and \eqref{eq: Linear Programming cone} where $\mathbb{A}$ is a $m\times n $ matrix of rank $m$,
	\begin{itemize}
		\item if there is a feasible solution, there is a basic feasible solution.
		\item if there is an optimal solution, there is an optimal basic feasible solution. 
	\end{itemize}
\end{theorem}
Since for a problem having $n$ variables and $m$ constraints there are at most
\begin{equation*}
	\binom{n}{m}=\frac{n!}{m!\parentheses{n-m}!}
\end{equation*}
basic solutions, the fundamental theorem of linear programming simplifies the problem to a finite number of possibilities. This is a  powerful theoretical result, but practical represents an inefficient method to find an optimal solution. This result has an interesting connection to convexity 

\begin{theorem}
	Let $\mathbf{A}$ be an $m\times n$ matrix of rank $m$ and $\mathbf{b}$ an $m$-vector. Let K be the convex polytope consisting of all $n$-vectors $\mathbf{x}$ satisfying
	\begin{align}
		\begin{array}{cc}
		\mathbf{A}\mathbf{x}&=\mathbf{b} \\
		\mathbf{x}&\geq 0		
		\end{array}
	\label{eq: Linear Programming Constrains and cone}
	\end{align}
	A vector $\mathbf{x}$ is an extreme point of $K$ if and only if $\mathbf{x}$ is a basic feasible solution of \eqref{eq: Linear Programming Constrains and cone}.
\end{theorem}
\begin{corollary}
If the convex set K corresponding to \eqref{eq: Linear Programming Constrains and cone} is nonempty, it has at least one extreme point.
\end{corollary}
\begin{corollary}
If there is a finite optimal solution to a linear programming problem, there is a finite optimal solution which is an extreme point of the constraint set.
\end{corollary}
\begin{corollary}
The constraint set $K$ corresponding to \eqref{eq: Linear Programming Constrains and cone} possesses at most a finite number of extreme points.
	\begin{proof}
	There is only a finite number of basic solutions generated by selecting $m$ basis vectors and $n$ columns of $\mathbf{A}$. The extreme points of $K$ are a subset of the basic solutions.
	\end{proof}
\end{corollary}

\begin{corollary}
	If the convex polytope $K$ corresponding to \eqref{eq: Linear Programming Constrains and cone} is bounded, then $K$ is a convex polyhedron. That is, $K$ consists of points that are convex combinations of a finite number of points.
\end{corollary}
\section{Duality}

\begin{lemma}[Weak Duality lemma]
If $\mathbf{x}$ and $\mathbf{v}$ are feasible for  and , respectively, then c T \textbf{xxxx} xT b.
\end{lemma}
\section{Simplex Method.}

\section{Interior Methods.}
