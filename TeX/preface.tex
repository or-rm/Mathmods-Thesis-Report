\chapter*{Preface}
\setheader{Preface}

The optimal transport problem was proposed by Gaspard Monge in 1781. Monge was interested in finding a way to transport a fixed amount of sand from one place into another, without losing or gaining mass in the process, such that the transportation cost is optimal. Later, Leonid Kantorovich introduced the proper framework to understand the problem in terms of probability measures, instead of using maps as Monge proposed the first time. The following work is a brief review of the theory developed for this problem, computation of a solution for a discrete optimal transport problem and two results of the developed theory that can be used in applications. 

The literature in this topic is really rich and researchers in empirical sciences (such as economy, physics, meteorology, among many others) have found connection between the optimal transport problem and problems related to their respective branches, that at first glance seem disconnected from the transportation problem. 

We have divided this text into five chapters. The first one is a reminder about concepts in topology, functional analysis, convex analysis and convergence of probability measures. These concepts are used to find the conditions that assure the existence of a minimizer for the optimal transport problem. If the reader is already familiarized with these concepts please skip directly to the third chapter. The second chapter is a reminder on important results in Linear programming, they are useful to compute a solution for the discrete version of the problem. In case the reader is already familiarized with linear programming, please skip directly to the fourth chapter. The third chapter is the heart of the text, it introduces the problem and we do a brief summary of th theory developed for the problem. The fourth chapter presents the algorithms used to compute a solution for a discrete version of the optimal transport. Finally, in the fifth chapter we discuss how the theory can be used as statistical distance between two probability measures, and an example of how a data assimilation problem can be also seen as an optimal transport problem.

\begin{flushright}
{\makeatletter\itshape
    \@author \\
    Barcelona, September 2018
\makeatother}
\end{flushright}

