\chapter{Preliminaries.}
We start this chapter reminding the basic definitions and theorems in topology and measure theory, since they are needed to have a suitable framework to discuss the optimal transport problem and its applications.

 
\section{Definitions and important theorems to remember.}

\subsection{Topology.}
We start with the definition of topology that is needed to introduce a notion of continuity. We refer to \cite{munkres2000topology} for more details in topology.

\begin{definition}[Topology]
	A topology on a set $X$ is a collection $\mathcal T$ of subsets of $X$ having the following properties
	\begin{itemize}
		\item  The space $X$ itself and $\emptyset$ are in $\mathcal T$.
		\item  The union of the elements of any sub-collection of $\mathcal T$ is in $\mathcal T$.
		\item The intersection of the elements of any finite sub-collection of $\mathcal T$ is in $\mathcal T$.
	\end{itemize}
\end{definition}

A pair $\parentheses{X, \mathcal{T}}$ is called a topological space. The elements of $\mathcal T$ are called open sets. The complements of the sets of $\mathcal T$ are called closed sets. The interior of a set $A$, is defined as the biggest open set contained in $A$. Similarly, the closure of a set $A$, is defined as the smallest closed set containing $A$. We use indistinctly the notation $\inter(A)$ and $A^{\circ}$ for the interior of a set $A$. In the same way, for the closure we use the notation $\closu(A)$ or $\bar{A}$. An equivalent way to define the same ideas is give by the following,

\begin{align*}
	\inter(A)=\bigcup_{B \text{ is open.}} B \\
	\closu(A)=\bigcap_{B \text{ is closed.}} B
\end{align*}

We remark that a set is open if and only if $A=A^{\circ}$, and a set is closed if and only if $A=\bar{A}$.  


We call a neighborhood of $x$ an element of $\mathcal{T}$ containing $x$.
\begin{definition}[Topological Basis.]
	Give a set $X$ endowed with a topology $\mathcal T$. We call a basis for $\mathcal{T}$ is a collection $\mathcal{B}$ of subsets of $X$ (called basis elements), such that,
	\begin{enumerate}
		\item  For each $x\in X$, there is at least one basis element $B$ containing $x$. 
		\item  If $x$ belongs to the intersection of two basis elements $B_1$ and $B_2$, then there is a basis element $B_3$ containing $x$ such that $B_3 \in B_1 \cap B_2$. 
	\end{enumerate}
\end{definition}



\begin{definition}[Dense set]
	A subset $D$ of a topological space $X$ is dense in $X$ if for any point $x$ in $X$, any neighborhood of $x$ contains at least one point from $D$.
\end{definition}

\begin{definition}[Separable space]
	A topological space is called separable if it contains a countable, dense subset.
\end{definition}

Topologies in which one element is not a closed set, or in which a sequence can converge to more than one point, are not really interesting for practical problems. If such things are allowed the theorems that one can prove are limited. A mathematician Felix Hausdorff suggested to add the following condition:
  
\begin{definition}[Hausdorff space]
A topological space $X$ is called a Hausdorff space if for each pair $x_1$, $x_2$ of distinct points of $X$, there exist neighborhoods $U_1$, and $U_2$ of $x_1$ and $x_2$, respectively, that are disjoint. 
\end{definition}

\begin{definition}[Distance]
\end{definition}
\begin{definition}[Metric Spaces]
	
\end{definition}

\begin{definition}[Completeness]
	A metric space $X$ is called complete if every Cauchy-Sequence of points in $X$ has a limit that is also in $X$. 
\end{definition}

\begin{definition}[Completely metrizable space]
	
\end{definition}
There is a subtle difference between complete metric space and completely metrizable space.  And the difference lies on the words ``\textit{there exists at least a metric...}" in the completely metrizable definition, and ``\textit{given a metric}''. Complete metrizable is a topological property while completeness is a property of the chosen metric.
\begin{definition}[Polish space]
	We call Polish space to any topological space that is separable and completely metrizable.
\end{definition}

\begin{definition}[Sequentially compact]
	A subset $K$ of a metric space $X$ is said to be compact if from any sequence $x_n$, we can extract a converging subsequence $x_{n_k} \rightarrow x \in K$.
\end{definition}

\begin{definition}[Compactness]
		A subset $K$ of a metric space $X$ is compact if every open cover of $K$ has a finite subcover. 
\end{definition}

\begin{theorem}
	A subset of a metric space is compact if and only if it is sequentially compact.
\end{theorem}


\begin{definition}[Liminf and Limsup]
	Let $X$ be a Hausdorff space. Let $\mathcal{V}(x_0)$ be a topological basis of $X$, such that all $V\in \mathcal{V}$ contains $x_0$. Let $f: X\rightarrow \Realex$ a functional valued in $\Realex$. We define, 
	\begin{equation*}
	\liminf_{x\rightarrow x_0} f(x)= \sup_{V\in \mathcal{V}(x_0)} \inf_{s\in V}\ f(s)
	\end{equation*}
	\begin{equation*}
	\limsup_{x\rightarrow x_0} f(x)= \inf_{V\in \mathcal{V}(x_0)} \sup_{s\in V}\ f(s)
	\end{equation*}
	The above definitions can be expressed in terms of sequences of real numbers. Let $(x_n)_{n \in \Naturals}$ be a sequence in $X$, the above formulation is equivalent to say.
	\begin{equation*}
	\liminf_{n\in \Naturals} x_n := \lim_{n\rightarrow \infty} \parentheses{\inf_{m\geq n} x_m}
	\end{equation*}
	Equivalently for $\limsup$,
	\begin{equation*}
	\liminf_{n\in \Naturals} x_n := \lim_{n\rightarrow \infty} \parentheses{\sup_{m\geq n} x_m}
	\end{equation*}
	Please note that the convergence to some point $x_0$, $(x_n)_{n\in \Naturals}\rightarrow x_0$ is not required in the last definitions.
\end{definition}


\subsection{Functional Analysis.}
%%%%%%%%%%%%%%%%%%%%%%%%%%%%%%%%%%Functional Analysis %%%%%%%%%%%%%%%%%%%%%%%%%%%%%%%%%%%%%%%%%%%%%%%%%%%%%%%%%%%%%%




\begin{definition}[Linear Space]
\end{definition}

\begin{definition}[Banach Space]
\end{definition}

\begin{definition}[Inner product]
\end{definition}

\begin{definition}[Hilbert Space]
\end{definition}

\begin{definition}[Continuity]
	
\end{definition}


For a given be a metric space $X$ . We denote the set of continuous, real-valued functions $f:X\rightarrow \Real$ by $C(X)$.

\begin{theorem}
	Let $K\subset X$ a compact subset of a metric space $X$. The space $C(K)$ is complete.
\end{theorem}
A natural norm on spaces of continuous functions is the uniform norm (also called infinity norm), which is defined by,
\begin{equation*}
\norm{f}_{\infty}=\sup_{x\in X} \abs{f(x)}
\end{equation*}
The norm $\norm{f}_{\infty}$ is finite if and only if $f$ is bounded. And we use $C_b(X)$ to refer the space of bounded functions on $X$.

\begin{definition}
	Let $f$ be a real valued function, $f:X\rightarrow \Real$ on a metric space. The \textbf{support of a function}, $\supp f$ is the closure of the set on which $f$ is nonzero.
	\begin{equation*}
	\supp f = \closu\parentheses{{\braces{x\in X: \quad f(x)\neq 0}}}
	\end{equation*}
	
\end{definition}
We say that $f$ has compact support if $\supp f$ is a compact subset of $X$, and denote the space of continuous functions on $X$ with compact support by $C_c(X)$.

The space $C_c(X)$ is a linear subspace of $C_b(X)$, but it does not need to be closed. 

\begin{definition}
	Suppose that X is a separable and locally compact metric space.	We say that a real valued function $f$ belongs to $C_0(X)$ if and only if $f\in C(X)$, and for every $\epsilon >0$, there exists a compact set $K\subset X$ such that $\abs{f}<\epsilon$ on $X\backslash K$. 
\end{definition}

\begin{definition}
	Let $\mathcal F$ be a family of functions from a metric space $\parentheses{X, d}$ to a metric space $\parentheses{Y, d}$. The family $\mathcal{F}$ is equicontinuous if for every $x\in X$ and $\epsilon > 0$ there is $\delta >0$ such that $d(x,y)<\delta$ implies $d(f(x), f(y))<\epsilon$ for all $f\in \mathcal F$.
\end{definition}

\begin{theorem}
	An equicontinuous family of functions from a compact metric space to a metric space is uniformly equicontinuous. 
\end{theorem}

\begin{theorem}[Ascoli-Arzel\`a]
	Let $K$ be a compact metric space. A subset $M$ of the set of continuous functions $M\subset C(K)$ is compact if and only if it is closed, bounded and equicontinuous. That is any sequence $(f_n)_{n\in\Naturals}$ in $M$ admits a subsequence converging to $f$ in $M$.
\end{theorem} 


\begin{definition}[Proper convex function]
	Let $f: X\rightarrow\Realex$, a function taking values in the extended real number line. We call it proper convex function if $\exists x \in X$ such that $f(x)<\infty$. And $\forall x \in X$, $f(x)>-\infty$.
\end{definition}

\begin{definition}[Projection of a Cartesian Product.]
	Let $\proj_x: X\times Y \rightarrow X$ be defined by the equation
	\begin{equation*}
		\proj_x(x,y)= x;
	\end{equation*}
	
	Equivalently, let $\proj_y: X\times Y \rightarrow Y$ be defined by,
	\begin{equation*}
		\proj_y(x,y)=y
	\end{equation*}
	These maps $\proj_x$ and $\proj_y$ are called the projections of $X\times Y$ onto $X$ and $Y$ respectively.	
\end{definition}

We can generalize a definition over a general Cartesian product. Given a set $X$, we define a \textit{$J$-tuple} of elements of $X$ to be a function $\mathbf{x}:J\rightarrow X$. If $\alpha$ is an element of $J$, we often denote the value of $\mathbf{x}$ at $\alpha$ by $x_\alpha$ rather than $\mathbf{x}(\alpha)$; we call it the $\alpha$-th coordinate of $\mathbf{x}$. And we often denote the function $\mathbf{x}$ by the symbol.

\begin{equation*}
	(x_\alpha)_{\alpha \in J}
\end{equation*} 

Let $\braces{A_\alpha}_{\alpha \in J}$ be a set of indexed family of sets; let $X=\bigcup_{\alpha \in J}A_{\alpha}$. The \textit{Cartesian product} of this indexed family, denoted by
\begin{equation*}
	\prod_{\alpha \in J}A_\alpha
\end{equation*}	
is defined to be the set of all $J$-tuples $\parentheses{x_\alpha}_{\alpha \in J}$ of elements of $X$ such that $x_\alpha \in A_\alpha$ for each $\alpha \in J$. That is, it is the set of all functions,
\begin{equation*}
\mathbf{x}: J\rightarrow \bigcup_{\alpha \in J} A_\alpha
\end{equation*}
such that $\mathbf{x}(\alpha)\in A_\alpha$ for each $\alpha \in J$.


\begin{definition}[Lower Semicontinuity]
	On a complete metric space $X$, a function $f: X\rightarrow\Realex$ is said to be lower semi-continuous (l.s.c.) if for every sequence $\parentheses{x_n}_{n\in\Naturals}$ converging to $x\in X$, we have 
	\begin{equation*}
	f(x)\leq \liminf_{n\in \Naturals} f(x_n)
	\end{equation*}
\end{definition}

We can see from the above definition that any continuous function is lower-semicontinuous. In other words, lower-semicontinuity is a milder requirement than continuity, although it preserves interesting properties that can be exploited in optimization.  

\begin{proposition}
	Let $f: X \rightarrow \Realex$ be a convex and lower-semicontinuous function.
	Assume that there exists $x_0 \in X$ such that $f(x_0) = -\infty$. Then $f$ is nowhere finite on $X$.
\end{proposition} 


\begin{theorem} If $ f_\alpha $ is an arbitrary family of lower semi-continuous functions on $X$, then $f=\sup_\alpha f_\alpha$ is also lower-semicontinuous.
\end{theorem}


\begin{definition}[Lipschitz condition]

\end{definition}

\begin{theorem}
	Let $f:X\rightarrow \Realex\backslash{-\infty}$ be a function bounded from below. Then $f$ is l.s.c. if and only if there exists a sequence $(f_n)_{n\in \Naturals}$ of $k$-Lipschitz functions such that for every $x\in X$, $f_n(x)$ converges increasingly to $f(x)$.
\end{theorem}

\begin{definition}
	Let $F:X\rightarrow \Realex$ be a given functional bounded from below on a metric space $X$.  Let $\mathcal G$ be the set of lower semicontinuous functions $G: X\rightarrow \Realex$, such that $G\leq F$. We call a relaxation the supremum of $\mathcal G$.  This functional does exist since the supremum of an arbitrary family of lower semicontinuous functions is also lower semicontinuous. It is possible to have a representation formula as follows:
	\begin{equation}
	\bar F(x) = \inf \braces{\liminf_{n\in \Naturals} F(x_n):\ x_n\rightarrow x}.
	\end{equation}
	As consequence of this definition we see that $F \geq \bar F$ implies $\inf F \geq \inf \bar F$. Let $l=\inf F$ then $F\geq l$. A constant function is lower semicontinuous. Therefore, $\bar F \geq l$ and $\inf \bar F \geq \inf F$. Implying that the infimum of both $F$ and its regularization $\bar F$ coincide, i.e. $\inf \bar F = \inf F$. 
\end{definition}


\begin{theorem}[Maxima and Minima]
	Let $X$ be a compact metric space and $f: X\rightarrow \Real$ is continuous, real-valued function. Then $f$ is bounded on $X$ and attains its maximum and minimum. That is, there are $x$, $y$ belonging to $X$ such that,
	\begin{equation*}
	f(x)=\inf_{z\in X}f(z) \qquad \mand \qquad f(y)=\sup_{z \in X} f(z)
	\end{equation*}
\end{theorem}

Continuity is a strong requirement. Luckily, we can assure the existence of a minimizer of lower-semicontinuous functionals (or maximizer for upper-semicontinuity). The usual procedure to prove existence of a minimizer is making use of Weierstrass' criterion. We take a minimizing sequence and then we prove that the space in which we are trying to find a minimizer element is compact.

\begin{theorem}[Weierstrass' criterion for existence of minimizers]
	If $f: X\rightarrow\Realex$ is lower semi-continuous and $X$ is compact, then there exists $\hat x \in X$ such $f(\hat x)=\min\braces{f(x):\ x\in X}$.
	\begin{proof}
		Define $l:=\inf\braces{f(x): \, x \in X} \in \Realex$, notice that $l=+\infty$ only if $f$ is identically $+\infty$, then this case is trivial since any point minimizes $f$. By compactness there exists a minimizing sequence $x_n$, that is $f(x_n)\rightarrow l$. By compactness we can extract a subsequence converging to some $\hat x$ such that $\hat x \in X$. By lower-semicontinuity of $f$, we have that $f(\hat x)\liminf_n f(x_n)=l$. Since $l$ is the infimum $l\leq f(\hat x)$. This proves that $l=f(\hat{x}) \in \Real$.
	\end{proof}
\end{theorem}

We can apply the above analysis using a notion of upper-semicontinuity and compactness to find the maximum.

\begin{definition}[Topological Dual]
	If $X$ is a normed space, the dual space $\asterisk{X}=\mathcal B(X, \Real)$. Consists of all linear and bounded functionals mapping from $X$ to $\Real$.
\end{definition}

\begin{definition}[Weak compactness in dual spaces]
	A sequence $x_n$ in a Banach space $X$ is said to be weakly converging to $x$, and we write $x_n \rightharpoonup x$, if for every $\xi \in \asterisk{X}$. We have $\anglesbox{\xi, x_n}\rightarrow \anglesbox{\xi, x}$. A sequence $\xi_n \in \asterisk{X}$ is said to be weakly-* converging to $\xi$, and we write $\xi_n \rightharpoonup \xi$, if for every $x\in X$ we have $\anglesbox{\xi_n, x} \rightarrow \anglesbox{\xi, n}$.
\end{definition}

\begin{theorem}[Banach-Alaouglu]
	If $X$ is separable and $\xi_n$ is a bounded sequence in $\asterisk{X}$, then there exists a subsequence $\xi_{n_k}$ weakly converging to some $\xi \in \asterisk{X}$.
\end{theorem}

The Banach-Alaouglu's theorem is a well known result in functional analysis, an equivalent formulation is saying the closed unit ball in $\asterisk{X}$ is weak-* compact.

%%%%%%%%%%%%%%%%%%%%%%%%%%%%%%%%%%%%%%%%%%%%%%%% Measure Theory %%%%%%%%%%%%%%%%%%%%%%%%%%%%%%%%%%%%%%%%%%%%%%%%%%%%%%%%%%%
\subsection{Measure Theory}

The optimal transport problem theory is based mostly on Measure Theory. We present some abstract objects and theorems needed to develop in proper way the problem. For better understanding on Measure Theory we refer \cite{bogachev2007Measure}. 
\begin{definition}[Sigma Algebra]
	An algebra of sets $\mathcal{A}$ is a class of subsets of some fixed set $X$ (called the space) such that,
	\begin{itemize}
		\item $X$ and $\emptyset$ belong to $\mathcal{A}$.
		\item If $A, B \in \mathcal{A}$, then $A\cap B \in \mathcal{A}$, $A\cup B \in \mathcal A$, $A\backslash B \in \mathcal A$.
	\end{itemize}
	An algebra of sets is a $\sigma$-algebra if for any sequence of sets $A_n \in \mathcal A$ we have $\mathcal{A}\ni \bigcup_{n\in \Naturals} A_n $.
\end{definition}

\begin{definition}[Measure Space] 
	A pair $(X, \mathcal A)$ consisting of a set $X$ and a $\sigma$-algebra $\mathcal A$ of its subsets is called a measurable space.
\end{definition}

\begin{definition}[Borel $\sigma$-algebra]
	The Borel $\sigma$-algebra $\mathcal{B}(\Real^n)$ of $\Real^n$ is the $\sigma$-algebra generated by all open sets. The sets in a Borel $\sigma$-algebra are called Borel sets. For any set $E\subset \Real^n$, let $\mathcal B(E)$ denote the class of all sets of the form $E\cap B$, where $B\in\mathcal B(\Real^n)$.
\end{definition}

Given a topological space, in this text consider the Borel $\sigma$-algebra unless stated otherwise. 

\begin{definition}[Measure]
	A real-valued set function $\mu:\mathcal{A}\rightarrow \Realex$ on a class of sets $\mathcal A$ is called countably additive if 
	\begin{equation*}
		\mu\parentheses{\bigcup_{n=1}^{\infty} A_n}=\sum_{n=1}^{\infty}\mu(A_n)
	\end{equation*}
	for all pairwise disjoint sets $A_n$ in $\mathcal A$ such that $\mathcal A \ni \bigcup_{n=1}^{\infty}A_i$. A countably additive set function defined on an algebra is called a measure.
\end{definition}

Given a measure space $X$, we denote $\MeasureSp(X)$ and $\PMeasureSp(X)$ to refer the set of finite measures and positive finite measures on $X$, respectively. 

\begin{definition}
	A countably additive measure $\mu$ on a $\sigma$-algebra of subsets of a space $X$ is called a probability measure if $\mu\geq 0$ and $\mu(X)=1$.
	A triple $(X, \mathcal A, \mu)$ is called a measure space if $\mu$ is a
	nonnegative measure on a $\sigma$-algebra A of subset of a set X. If $\mu$ is a probability measure, then $(X, \mathcal A, \mu)$ is called a probability space. 
\end{definition}


\begin{definition}[Lebesgue Measure]
\end{definition}

\begin{definition}[Hausdorff Measure]
\end{definition}

\begin{definition}[Probability]
	A countably additive measure $\mu$ on a $\sigma$-algebra of subsets of a space $X$ is called a probability measure if $\mu \geq 0$ and $\mu(X)= 1$.
	
	A triple $(X, \mathcal A, \mu )$ is called a measure space if $\mu$ is a
	nonnegative measure on a $\sigma$-algebra $\mathcal A$ of subset of a set $X$. If $\mu$ is a probability measure, then $(X, \mathcal A, \mu)$ is called a probability space.
\end{definition}

\begin{definition}[Product $\sigma$-algebra ]
Let $\left(X_1 , \mathcal{A}_1 , \mu_1 \right)$ and $\left(X_1 , \mathcal{A}_1 , \mu_1 \right)$ be two spaces with finite nonnegative measures. On the space $X_1\times X_2$ we consider sets of the form $A_1\times A_2$, where $A_i \in \mathcal{A}_i$, called measurable rectangles. Let $\mu_1\times \mu_2 \parentheses{A_1\times A_2}:= \mu_1(A_1)\mu_2(A_2)$. 

Let $A_1\otimes A_2$ denote the $\sigma$-algebra generated by all measurable rectangles; this $\sigma$-algebra is called the product of the $\sigma$-algebras $A_1$ and $A_2$.  
\end{definition}

\begin{theorem}
The set function $\mu_1\times \mu_2$ is countably additive on the algebra generated by all measurable rectangles and uniquely extends to a countably additive measure, denoted by $\mu_1\otimes \mu_2$.
\end{theorem}

\begin{definition}[Image Measure]
	Let $\parentheses{X, \mathcal{A}_X}$  and $\parentheses{Y, \mathcal{A}_Y}$ be two measurable spaces. Let $T:X\rightarrow Y$ be a measurable map from $X$ to $Y$. Let $\mu$ be a measure $\mu: \mathcal{A}_X\rightarrow \Realex_+$, then the image measure (or pushforward measure) $\Tmeasure{\mu}{T}:\mathcal{A}_2\rightarrow \Realex_+$ is given by,
	\begin{equation*}
		\Tmeasure{\mu}{T}(B) = \mu\parentheses{T^-1(B)}, \quad \forall B \in \mathcal{A}_Y.
	\end{equation*}
\end{definition}


\begin{theorem}
	Let $\parentheses{X, \mathcal{A}_X}$  and $\parentheses{Y, \mathcal{A}_Y}$ be two measurable spaces. Let $\mu$ be a nonnegative measure. A $\mathcal{A}_2$-measurable function $g$ on $Y$ is integrable with respect the measure $\mu\circ f^{-1}$ precisely when the function $g\circ f$ is integrable with respect to $\mu$. In addition we have, 
	\begin{equation*}
		\int_Y g(y)\mu\circ f^{-1}(\dy) = \int_X g(f(x))\mu(\dx)
	\end{equation*}
\end{theorem}

The space of Borel probability measures on $X$ is denoted by $\PlanSp(X)$. The weak topology on $\PlanSp(X)$ is induced by convergence against bounded continuous test functions on $X$, that is $C_b(X)$.

\begin{definition}[Atom and atomless measures]
	The set $A \in \mathcal A$ is called an atom of the measure $\mu$
	if $\mu(A) > 0$ and every set $B \subset A$ from $\mathcal A$ has measure either $0$ or $\mu(A)$. If there are no atoms, then the measure $\mu$ is called atomless. 
\end{definition}

A measure over a set $\Omega\subset \Real$ is atomless if $\forall x \in \Omega$, we have  $\mu(\{x\})=0$. The Dirac's measure is not atomless. 
\begin{definition}[Absolutely continuity and singularity]
	Let $\mu$ and $\nu$ be countably additive measures on a measurable space $\parentheses{X, \mathcal A}$. 
	\begin{itemize}
		\item The measure $\nu$ is called absolutely continuous with respect to $\mu$ if $\abs{\nu}(A)=0$ for every set $A$ with $\abs{\mu}=0$. We use the notation $\nu \ll \mu$.
		\item The measure $\nu$ is called singular with respect to $\mu$ if there exists a set $A\in \mathcal{A}$ such that
		\begin{equation*}
			\abs{\mu}(A)=0 \quad \text{ and }\quad  \abs{v}(X\backslash A)=0
		\end{equation*}
	\end{itemize}
	
	If $\nu \ll \mu $ and $\mu \ll \nu$, then the measures $\mu$ and $\nu$ are equivalent. We use the notation $\mu \sim \nu$ to refer this situation.
\end{definition}
The above definition allows us to introduce the Radon-Nikodym theorem that is one of the main results in measure theory.
\begin{theorem}[Radon–Nikodym theorem]
	Let $\mu$ and $\nu$ be two finite measures on a space $(X, \mathcal A)$.
	The measure $\nu$ is absolutely continuous with respect to the measure $\mu$ precisely	when there exists a $\mu$-integrable function $f$ such that $\nu$ is given by
	\begin{equation*}
		v(A)=\int_A f\dmu
	\end{equation*}
\end{theorem}

\begin{definition}[$L^p$ Spaces]
	
\end{definition}



\begin{theorem}[Lebesgue dominated convergence theorem]
	Suppose that $\mu$-integrable functions $f_n$ converge almost everywhere to a function $f$. If there exists a $\mu$-integrable function $\Phi$ such that,
	\begin{equation*}
		\abs{f_n}(x)\leq \Phi (x), \quad \text{almost everywhere for every } n
	\end{equation*}
	then the function is integrable and
	\begin{equation*}
		\int_X f(x)\dmu(x) = \lim_{n\rightarrow\infty} \int_X f_n(x)\dmu(x)
	\end{equation*}
	In addition,
	\begin{equation*}
		\lim_{n\rightarrow \infty } \int \abs{f(x)-f_n(x)}\dmu(x) =0
	\end{equation*}
\end{theorem}

\begin{theorem}[Monotone Convergence]
	Let $(f_n)_{n\in \Naturals}$ be a sequence of $\mu$-integrable functions such that $f_n(x) \leq f_{n+1}$ almost everywhere for each $n \in \Naturals$. Suppose that
	\begin{equation}
	\sup_{n\in \Naturals}\int_X f_n(x)\dmu(x) < \infty
	\end{equation}
	
	Then the function $f(x)=\lim_{n\rightarrow\infty} f_n(x)$ is almost everywhere finite and integrable. In addition the following equality holds,
	\begin{equation*}
		\int_X f(x)\dmu(x) = \lim_{n\rightarrow\infty} \int_X f_n(x)\dmu(x)	
	\end{equation*}
\end{theorem}

\begin{theorem}[Fatou's Theorem]
	Let $(f_n)_{n\in \Naturals}$ be a sequence of nonnegative $\mu$-integrable functions convergent to a function $f$ almost everywhere and let
	\begin{equation*}
		\sup_{n\in \Naturals} \int_X f_n(x)\dmu(x)\leq K <\infty
	\end{equation*}
	
	Then, the function $f$ is $\mu$-integrable and 
	\begin{equation*}
		\int_X f(x)\dmu(x) \leq K
	\end{equation*}
	
	Moreover, 
	\begin{equation*}
		\int_X f(x)\dmu(x)\leq \liminf_{n\rightarrow\infty} \int_{X} f_n (x)\dmu(x)
	\end{equation*}
\end{theorem}

\begin{corollary}
	Let $\parentheses{f_n}_{n\in \Naturals}$ be a sequence of nonnegative $\mu$-integrable functions such that 
	\begin{equation*}
		\sup_{n\in \Naturals}\int_{X} f_n(x)\dmu(x) \leq K < \infty
	\end{equation*}
	Then the function $\liminf_{n\rightarrow \infty} f_n$ is $\mu$-integrable and one has 
	\begin{equation}
		\int_X \liminf_{n\rightarrow\infty}f_n(x)\dmu(x)\leq \liminf_{n\rightarrow \infty} \int_X f_n(x)\dmu(x) \leq K
	\end{equation}
\end{corollary}

\begin{corollary}
The dominated convergence theorem and Fatou’s theorem remain valid if in place of almost everywhere convergence in their hypotheses we require convergence of $(f_n)_{n\in \Naturals}$ to $f$ in measure $\mu$.
\end{corollary}



\begin{definition}[Tightness] Let $(X, \mathcal T)$ a topological space, and let $\mathcal A$ a $\sigma$-algebra on $X$ that contains the topology $\mathcal{T}$. Let $M$ be a collection of measures defined on $\mathcal{A}$. The collection $M$ is called tight  if for every $\epsilon>0$ there is a compact subset $K_\epsilon$ of $X$ such that, for all measures $\mu\in M$ we have,
	\begin{equation*}
	\abs{\mu}\parentheses{X \backslash K_\epsilon} < \epsilon
	\end{equation*} 
\end{definition}


\begin{definition}
	A sequence $\mu_n$ probability measures over $X$ is said to be tight if for every $\epsilon> 0$, there exists a compact subset $K\subset X$ such that $\mu_n\parentheses{X\backslash K}< \epsilon$ for every $n$.
\end{definition}


\begin{theorem}[Prokhorov]
	Suppose that $\mu_n $ is a tight sequence of probability measures over a Polish space $X$. Then there exists $\mu\in\PlanSp(X)$ and a subsequence $\mu_{n_k}$ such that $\mu_{n_k}\rightharpoonup \mu $, in duality with $C_b(X)$. Conversely, every sequence $\mu_{n_k}\rightharpoonup \mu$ is tight.
\end{theorem}

\begin{definition}
	Let $\parentheses{X, \mathcal A, \mu}$ be a probability space. Then every Borel-measurable mapping $\X: X\rightarrow \Real$ with for all $B\in \mathcal B(\Real) $ is a random variable, denoted by $\X: \parentheses{X, \mathcal A}\rightarrow \parentheses{\Real, \mathcal B (\Real)} $
\end{definition}

\begin{definition}[Duality between $C_0$ and $\MeasureSp$ ]
		
\end{definition}


\textbf{Explanation about notions of convergence with bounded functionals and vanishing in infinity functions}. If $X$ is compact we have $C_0(X)=C_b(X)=C(X)$ if $X$ and both notions of convergence coincide.

\begin{theorem}[Rademacher]
	Let $f:\Real^d\rightarrow\Real$ be a Lipschitz continuous function. Then the set of points where $f$ is not differentiable is negligible for the Lebesgue measure.
\end{theorem}


\begin{lemma}
If $\mu$, $\nu$ are two probability measures on the real line $\Real$ and $\mu$ is atomless, then there exists at least a map $T$ such that $\Tmeasure{\mu}{T}=\nu$.
\end{lemma}

\begin{lemma}
There exists a Borel map $\sigma_d: \Real^d\rightarrow \Real$ which is injective, its image is a Borel subset of $\Real$, and its inverse map is Borel measurable as well.
\end{lemma}

\begin{theorem}
If $\mu$ and $\nu$ are two probability measures on $\Real^d$ and $\mu$ is atomless, then there exists at least a map $T$ such that $\Tmeasure{\mu}{T}=\nu$.
\end{theorem}