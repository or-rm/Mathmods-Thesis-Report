\chapter{Preliminaries.}
\section{Definitions and important theorems to remember.}

We start this chapter reminding the basic definitions and theorems in topology and measure theory, since they are needed to have a suitable framework to discuss the optimal transport problem and its applications. 

We start with the definition of topology that is needed to introduce a notion of continuity. We refer to \cite{munkres2000topology} for more details.
\begin{definition}[Topology]
	A topology on a set $X$ is a collection $\mathcal T$ of subsets of $X$ having the following properties
	\begin{itemize}
		\item  The space $X$ itself and $\emptyset$ are in $\mathcal T$.
		\item  The union of the elements of any sub-collection of $\mathcal T$ is in $\mathcal T$.
		\item The intersection of the elements of any finite sub-collection of $\mathcal T$ is in $\mathcal T$.
	\end{itemize}
\end{definition}

A pair $\parentheses{X, \mathcal{T}}$ is called a topological space. The elements of $\mathcal T$ are called open sets. The complements of the sets of $\mathcal T$ are called closed sets. We call a neighborhood of $x$ an element of $\mathcal{T}$ containing $x$.

\begin{definition}[Topological Basis.]
	Give a set $X$ endowed with a topology $\mathcal T$. We call a basis for $\mathcal{T}$ is a collection $\mathcal{B}$ of subsets of $X$ (called basis elements), such that,
	\begin{enumerate}
		\item  For each $x\in X$, there is at least one basis element $B$ containing $x$. 
		\item  If $x$ belongs to the intersection of two basis elements $B_1$ and $B_2$, then there is a basis element $B_3$ containing $x$ such that $B_3 \in B_1 \cap B_2$. 
	\end{enumerate}
\end{definition}

\begin{definition}[Dense set]
	A subset $D$ of a topological space $X$ is dense in $X$ if for any point $x$ in $X$, any neighborhood of $x$ contains at least one point from $D$.
\end{definition}

\begin{definition}[Separable space]
	A topological space is called separable if it contains a countable, dense subset.
\end{definition}

\begin{definition}
	Let $X$ be a Hausdorff space. Let $\mathcal{V}(x_0)$ be a topological basis of $X$, such that all $V\in \mathcal{V}$ contains $x_0$. Let $f: X\rightarrow \Realex$ a functional valued in $\Realex$. We define, 
	\begin{equation*}
	\liminf_{x\rightarrow x_0} f(x)= \sup_{V\in \mathcal{V}(x_0)} \inf_{s\in V}\ f(s)
	\end{equation*}
	\begin{equation*}
	\limsup_{x\rightarrow x_0} f(x)= \inf_{V\in \mathcal{V}(x_0)} \sup_{s\in V}\ f(s)
	\end{equation*}
	
The above definitions can be expressed in terms of sequences of real numbers. Let $(x_n)_{n \in \Naturals}$ be a sequence in $X$, the above formulation is equivalent to say.
\begin{equation*}
	\liminf_{n\in \Naturals} x_n := \lim_{n\rightarrow \infty} \parentheses{\inf_{m\geq n} x_m}
\end{equation*}
Equivalently for $\limsup$,
	\begin{equation*}
	\liminf_{n\in \Naturals} x_n := \lim_{n\rightarrow \infty} \parentheses{\sup_{m\geq n} x_m}
	\end{equation*}

Please note that the convergence to some point $x_0$, $(x_n)_{n\in \Naturals}\rightarrow x_0$ is not required in the last definitions.
\end{definition}

\begin{definition}[Lower Semicontinuity]
	On a complete metric space $X$, a function $f: X\rightarrow\Realex$ is said to be lower semi-continuous (l.s.c.) if for every sequence $\parentheses{x_n}_{n\in\Naturals}$ converging to $x\in X$, we have \begin{equation*}
		f(x)\leq \liminf_{n\in \Naturals} f(x_n)
	\end{equation*}
\end{definition}

\begin{definition}[Sequentially compact]
	A subset $K$ of a metric space $X$ is said to be compact if from any sequence $x_n$, we can extract a converging subsequence $x_{n_k} \rightarrow x \in K$.
\end{definition}
We can see from the above definition that any continuous function is lower-semicontinuous. 

\begin{definition}[Compactness]
		A subset $K$ of a metric space $X$ is compact if every open cover of $K$ has a finite subcover. 
\end{definition}

\begin{theorem}
	A subset of a metric space is compact if and only if it is sequentially compact.
\end{theorem}

\begin{theorem}[Maxima and Minima]
	Let $X$ be a compact metric space and $f: X\rightarrow \Real$ is continuous, real-valued function. Then $f$ is bounded on $X$ and attains its maximum and minimum. That is, there are $x$, $y$ belonging to $X$ such that,
	\begin{equation*}
		f(x)=\inf_{z\in X}f(z) \qquad \mand \qquad f(y)=\sup_{z \in X} f(z)
	\end{equation*}
\end{theorem}
Continuity is a strong requirement. Luckily we can assure the existence of a minimizer on lower-semicontinuous functions (or maximizer on upper-semicontinuous). The usual procedure to prove existence of a minimizer is making use of Weierstrass' criterion. We take a minimizing sequence and then we prove that the space in which we are trying to find a minimizer element is compact.
\begin{theorem}[Weierstrass' criterion for existence of minimizers]
	If $f: X\rightarrow\Realex$ is lower semi-continuous and $X$ is compact, then there exists $\hat x \in X$.
	\begin{proof}
		Define $l:=\inf\braces{f(x): \, x \in X} \in \Realex$, notice that $l=+\infty$ only if $f$ is identically $+\infty$, then this case is trivial since any point minimizes $f$. By compactness there exists a minimizing sequence $x_n$, that is $f(x_n)\rightarrow l$. By compactness we can extract a subsequence converging to some $\hat x$ such that $\hat x \in X$. By lower-semicontinuity of $f$, we have that $f(\hat x)\liminf_n f(x_n)=l$. Since $l$ is the infimum $l\leq f(\hat x)$. This proves that $l=f(\hat{x}) \in \Real$.
	\end{proof}
\end{theorem}

We can apply the above analysis using a notion of upper-semicontinuity and compactness to find the maximum. Explanation about notions of convergence with bounded functionals and vanishing in infinity functions. If $X$ is compact we have $C_0(X)=C_b(X)=C(X)$ if $X$ and both notions of convergence coincide.

\begin{definition}[Topological Dual]
	If $X$ is a normed space, the dual space $\asterisk{X}=\mathcal B(X, \Real)$. Consists of all linear and bounded functionals mapping from $X$ to $\Real$.
\end{definition}

\begin{definition}[Weak compactness in dual spaces]
	A sequence $x_n$ in a Banach space $X$ is said to be weakly converging to $x$, and we write $x_n \rightharpoonup x$, if for every $\xi \in \asterisk{X}$. We have $\anglesbox{\xi, x_n}\rightarrow \anglesbox{\xi, x}$. A sequence $\xi_n \in \asterisk{X}$ is said to be weakly-* converging to $\xi$, and we write $\xi_n \rightharpoonup \xi$, if for every $x\in X$ we have $\anglesbox{\xi_n, x} \rightarrow \anglesbox{\xi, n}$.
\end{definition}

\begin{theorem}[Banach-Alaouglu]
	If $X$ is separable and $\xi_n$ is a bounded sequence in $\asterisk{X}$, then there exists a subsequence $\xi_{n_k}$ weakly converging to some $\xi \in \asterisk{X}$.
\end{theorem}
The Banach-Alaouglu's theorem is a well known result in functional analysis, an equivalent formulation is saying the closed unit ball in $\asterisk{X}$ is weak-* compact.

\begin{definition}
	A sequence $\mu_n$ probability measures over $X$ is said to be tight if for every
	$\epsilon> 0$, there exists a compact subset $K\subset X$S such that $\mu_n\parentheses{X\backslash K}< \epsilon$ for every $n$.
\end{definition}

\begin{definition}[Tightness] Let $(\X, \mathcal T)$ a topological space, and let $\mathcal A$ a $\sigma$-algebra on $\X$ that contains the topology $\mathcal{T}$. Let $M$ be a collection of measures defined on $\mathcal{A}$. The collection $M$ is called tight  if for every $\epsilon>0$ there is a compact subset $K_\epsilon$  of $\X$ such that, for all measures $\mu\in M$ we have,
	\begin{equation*}
		\abs{\mu}\parentheses{\X \backslash K_\epsilon} < \epsilon
	\end{equation*} 
\end{definition}


\begin{theorem}[Prokhorov]
	Suppose that $\mu_n $ is a tight sequence of probability measures over a Polish space $\X$. Then there exists $\mu\in\PlanSp(X)$ and a subsequence $\mu_{n_k}$ such that $\mu_{n_k}\rightharpoonup \mu $, in duality with $C_b(X)$. Conversely, every sequence $\mu_{n_k}\rightharpoonup \mu$ is tight.
\end{theorem}

\begin{definition}
	Let $F:X\rightarrow \Realex$ be a given functional bounded from below on a metric space $X$.  Let $\mathcal G$ be the set of lower semicontinuous functions $G: X\rightarrow \Realex$, such that $G\leq F$. We call a relaxation the supremum of $\mathcal G$.  This functional does exist since the supremum of an arbitrary family of lower semicontinuous functions is also lower semicontinuous. It is possible to have a representation formula as follows:
	\begin{equation}
		\bar F(x) = \inf \braces{\liminf_{n\in \Naturals} F(x_n):\ x_n\rightarrow x}.
	\end{equation}
	As consequence of this definition we see that $F \geq \bar F$ implies $\inf F \geq \inf \bar F$. Let $l=\inf F$ then $F\geq l$. A constant function is lower semicontinuous. Therefore, $\bar F \geq l$ and $\inf \bar F \geq \inf F$. Implying that the infimum of both $F$ and its regularization $\bar F$ coincide, i.e. $\inf \bar F = \inf F$. 
\end{definition}
\begin{definition}
	A measure over a set $\Omega$ is atomless if $\forall x \in \Omega$, we have  $\mu(\{x\})=0$.
\end{definition}
\begin{definition}[Proper convex function]
	Let $f: X\rightarrow\Realex$, a function taking values in the extended real number line. We call it proper convex function if $\exists x \in X$ such that 
	\begin{equation*}
		f(x)<\infty
	\end{equation*}
	And $\forall x \in X$,
	\begin{equation}
		f(x)>-\infty
	\end{equation}
\end{definition}
