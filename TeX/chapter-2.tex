\chapter{Basics in Convex Analysis.}

\begin{definition}[Convexity]
Let $X$ a linear space. 
\end{definition}
\begin{definition}[Graph and Epigraph]
\end{definition}
\begin{definition}[Infimal Convolution]
\end{definition}
\begin{definition}[Extreme Point]
A point $x$ in a convex set $C$ is said to be an extreme point of $C$ if there are no two distinct points $x_1$ and $x_2$ in $C$ such that $x = \alpha x_1 + \parentheses{1-\alpha}x_2$ for some $0<\alpha <1$.
\end{definition}

\begin{proposition}
Let $f: X \rightarrow \Realex$ be a convex and lower-semicontinuous function.
Assume that there exists $x_0 \in X$ such that $f(x_0) = -\infty$. Then $f$ is nowhere finite on $X$.
\end{proposition} 


\begin{theorem} If $ f_\alpha $ is an arbitrary family of lower semi-continuous functions on $X$, then $f=\sup_\alpha f_\alpha$ is also lower-semicontinuous.
\end{theorem}


\begin{definition}[Convex conjugate function]
Let $X$ be a Banach space, let $f:X\rightarrow \Realex$ be a functional over $X$. We call the convex conjugate to the function $\asterisk{f}: \asterisk{X}\rightarrow \Realex$, defined as
\begin{equation*}
	\asterisk{f}(\asterisk{x})=\sup_{x\in X} \braces{\anglesbox{\asterisk{x}, x}-f(x)}
\end{equation*}	
\end{definition}

\begin{proposition}
	The convex conjugate $\asterisk{f}:\asterisk{X}\rightarrow \Realex$ of a function $f:X\rightarrow\Realex$ is convex.
	\begin{proof}
		Let $\asterisk{x}, \asterisk{y}$ elements of the dual space $\asterisk{X}$, and $t\in[0,1]$,
		\begin{align*}
			\asterisk{f}\parentheses{t\asterisk{x}+(1-t)\asterisk{y}}&=\sup_{x\in X}\braces{\anglesbox{t\asterisk{x}+(1-t)\asterisk{y}, x}-f(x)} \\
			&=\sup_{x\in X}\braces{\anglesbox{t\asterisk{x}+(1-t)\asterisk{y}, x}-tf(x)-(1-t)f(x)}\\
			&=\sup_{x\in X}\braces{t\anglesbox{\asterisk{x}, x}+(1-t)\anglesbox{\asterisk{y}, x}-tf(x)-(1-t)f(x)}\\
			&\leq \sup_{x, y \in X} \braces{t\anglesbox{\asterisk{x}, x}+(1-t)\anglesbox{\asterisk{y}, y}-tf(x)-(1-t)f(y)}\\
			&=t\sup_{x \in X} \braces{\anglesbox{\asterisk{x}, x}-f(x)}+(1-t)t\sup_{y \in X} \braces{\anglesbox{\asterisk{y}, y}-f(y)}\\
			&=t\asterisk{f}(\asterisk{x})+(1-t)\asterisk{f}(\asterisk{y}).
		\end{align*}
		Therefore $\asterisk{f}$ is convex regardless the convexity of $f$.
	\end{proof}
\end{proposition}

\begin{theorem} 
	A function $f:\Real^d \rightarrow \Realex$ is convex and lower-semicontinuous if and only if $f^{**}=f$.
\end{theorem}
	
\begin{lemma}[Convex envelope theorem]
	Let $X$ be a reflexive Banach Space. Then the convex conjugate function $f^*$ is the maximum convex functional below $f$ (also called convex envelope), i.e. if $\varrho$ is convex functional and $\varrho(x)\leq f(x)$, $\forall x \in X$. Then, $f^{**}(x)\leq f(x)$, and $\varrho(x)\leq f^{**}(u)$, $\forall x \in U $. In particular $f^{**} = f$ if and only if J is convex.
\end{lemma}

\begin{definition}[Legendre Transform]
	Let $f: \Real^d\rightarrow \Realex$ be a convex function, we call the Legendre transform $f^*$
	\begin{equation*}
	f^*(y)=\sup_{x\in \Real^d} \braces{x\cdot y -f(x)}
	\end{equation*}
\end{definition}

\begin{corollary}
	A function $f:\Real^d\rightarrow\Realex$  is convex and l.s.c. if and only if $f^{∗∗}=f$.
\end{corollary}

\begin{definition}[Subdifferential]
	Given a proper convex function $f: X\rightarrow (-\infty, \infty]$, the subdifferential of such a function is the mapping $\partial f: X\rightarrow X^*$ defined by,
	\begin{equation*}
		\partial f (x)= \braces{x^*\in X^*; f(x)-f(y) \leq \anglesbox{x^*, x-y},\ \forall y \in X}
	\end{equation*}
\end{definition}

\begin{theorem}[Geometrical version of Hahn-Banach Theorem]
	Let $M$ be a vector subspace of the topological vector space X. Suppose $K$ is a non-empty convex open subset of $X$ with $K\cap M=\emptyset$. Then there is a closed hyperplane $N \in X$ containing M with $K \cap N = \emptyset$.
\end{theorem}

\begin{theorem}
	The epigraph of a convex and lower semicontinuous function is a closed convex set in $\Real^d\times\Real$, and can be written as the intersection of the half-spaces which contain it.
\end{theorem}
	
\begin{definition}[Projection onto a Set]
\end{definition}
\begin{theorem}
\end{theorem}


\begin{definition}[Duality]

\end{definition}