\chapter{Basics in Convex Analysis.}
\begin{definition}[Variety]
\end{definition}
\begin{definition}[Convexity]
Let $X$ be a linear space. 
\end{definition}
\begin{definition}[Graph and Epigraph]
\end{definition}
\begin{theorem}[Hahn-Banach separation theorem]
\end{theorem}

\begin{theorem}[Geometrical version of Hahn-Banach Theorem]
	Let $M$ be a vector subspace of the topological vector space X. Suppose $K$ is a non-empty convex open subset of $X$ with $K\cap M=\emptyset$. Then there is a closed hyperplane $N \in X$ containing M with $K \cap N = \emptyset$.
\end{theorem}

\begin{definition}[Cone]
	
\end{definition}


\begin{definition}[Extreme Point]
A point $x$ in a convex set $C$ is said to be an extreme point of $C$ if there are no two distinct points $x_1$ and $x_2$ in $C$ such that $x = \alpha x_1 + \parentheses{1-\alpha}x_2$ for some $0<\alpha <1$.
\end{definition}

\begin{definition}[Convex conjugate function]
Let $X$ be a Banach space, let $f:X\rightarrow \Realex$ be a functional over $X$. We call the convex conjugate to the function $\asterisk{f}: \asterisk{X}\rightarrow \Realex$, defined as
\begin{equation*}
	\asterisk{f}(\asterisk{x})=\sup_{x\in X} \braces{\anglesbox{\asterisk{x}, x}-f(x)}
\end{equation*}	
\end{definition}

\begin{proposition}
	The convex conjugate $\asterisk{f}:\asterisk{X}\rightarrow \Realex$ of a function $f:X\rightarrow\Realex$ is convex.
	\begin{proof}
		Let $\asterisk{x}, \asterisk{y}$ elements of the dual space $\asterisk{X}$, and $t\in[0,1]$,
		\begin{align*}
			\asterisk{f}\parentheses{t\asterisk{x}+(1-t)\asterisk{y}}&=\sup_{x\in X}\braces{\anglesbox{t\asterisk{x}+(1-t)\asterisk{y}, x}-f(x)} \\
			&=\sup_{x\in X}\braces{\anglesbox{t\asterisk{x}+(1-t)\asterisk{y}, x}-tf(x)-(1-t)f(x)}\\
			&=\sup_{x\in X}\braces{t\anglesbox{\asterisk{x}, x}+(1-t)\anglesbox{\asterisk{y}, x}-tf(x)-(1-t)f(x)}\\
			&\leq \sup_{x, y \in X} \braces{t\anglesbox{\asterisk{x}, x}+(1-t)\anglesbox{\asterisk{y}, y}-tf(x)-(1-t)f(y)}\\
			&=t\sup_{x \in X} \braces{\anglesbox{\asterisk{x}, x}-f(x)}+(1-t)t\sup_{y \in X} \braces{\anglesbox{\asterisk{y}, y}-f(y)}\\
			&=t\asterisk{f}(\asterisk{x})+(1-t)\asterisk{f}(\asterisk{y}).
		\end{align*}
		Therefore $\asterisk{f}$ is convex regardless the convexity of $f$.
	\end{proof}
\end{proposition}


	
\begin{theorem}[Convex envelope theorem]
	Let $X$ be a reflexive Banach Space. Then the convex conjugate function $f^*$ is the maximum convex functional below $f$ (also called convex envelope), i.e. if $\varrho$ is convex functional and $\varrho(x)\leq f(x)$, $\forall x \in X$. Then, $f^{**}(x)\leq f(x)$, and $\varrho(x)\leq f^{**}(u)$, $\forall x \in U $. In particular $f^{**} = f$ if and only if J is convex.
\end{theorem}

\begin{corollary} 
	A function $f:\Real^d \rightarrow \Realex$ is convex and lower-semicontinuous if and only if $f^{**}=f$.
\end{corollary}
\begin{definition}[Cyclically monotone]
A set $A\subset \Real^d\times\Real^d$ is said to be cyclically monotone if for every $N\in \Naturals$,  every permutation $\sigma$ and every finite family of points $\braces{(x_1, p_1), \dots, (x_N, p_N)}\in A$, we have
\begin{equation}
	\sum_{i=1}^{N}\anglesbox{x_i, p_i} \geq \sum_{i=1}^{N}\anglesbox{x_i, p_{\sigma_i}}.
\end{equation}
\end{definition}
\begin{definition}[Subdifferential]
	Given a proper convex function $f: X\rightarrow (-\infty, \infty]$, the subdifferential of such a function is the mapping $\partial f: X\rightarrow X^*$ defined by,
	\begin{equation*}
		\partial f (x)= \braces{x^*\in X^*; f(x)-f(y) \leq \anglesbox{x^*, x-y},\ \forall y \in X}
	\end{equation*}
\end{definition}

\begin{definition}[Gateaux derivative.]
\end{definition}
\begin{theorem}
	Any convex function is Gateaux-differentiable.
\end{theorem}
\begin{theorem}
	The epigraph of a convex and lower semicontinuous function is a closed convex set in $\Real^d\times\Real$, and can be written as the intersection of the half-spaces which contain it.
\end{theorem}
	\textbf{Here we write the proof for the identity for the projection onto an affine set}
\begin{definition}[Projection onto a Set]
\end{definition}
\begin{theorem}
\end{theorem}

An important concept in convex programming is duality.


\begin{definition}[Duality]

\end{definition}

\section{Convexity in $\Real^n$}
Since many results are developed in finite dimensions we state some results related to convexity in $\Real^n$.
\begin{theorem}
Let $C$ be a convex set and let $y$ be a point exterior to the closure of $C$. Then there is a vector $\mathbf{a}$ such that $\mathbf{a}^\top \mathbf{y} < \inf_{x\in C} \mathbf{a}^\top \mathbf{x}$. 
\end{theorem}

\begin{definition}
	A hyperplane in $\Real^n$ is an $(n-1)$-dimensional linear variety.
\end{definition}

\begin{proposition}
	Let $\mathbf{a}$ be a nonzero $n$-dimensional column vector, and let $c$ be a real number. The set
	\begin{equation*}
		H=\braces{\mathbf{x}\in \Real^n:\quad \mathbf{a}^\top \mathbf{x}=c}
	\end{equation*}
	is a hyperplane in $\Real^n$.
\end{proposition}

\begin{proposition}
	Let $H$ be a hyperplane in $\Real^n$. Then there is a nonzero $n$-dimensional vector and a constant $c$ such that,
	\begin{equation*}
		H=\braces{\mathbf{x}\in\Real^n:\quad \mathbf{a}^\top \mathbf{x}=c}
	\end{equation*}
\end{proposition}

\begin{definition}
	Let $\mathbf{a}$ be a nonzero vector in $\Real^n$ and let $c$ be a real number. Corresponding to the hyperplane $H=\braces{\mathbf{x}:\quad \mathbf{a}^\top \mathbf{x}=c}$ are the positive and negative closed half spaces 
	\begin{align*}
		H_+ = \braces{\mathbf{x}:\quad \mathbf{a}^\top \mathbf{x}\geq c}\\
		H_- = \braces{\mathbf{x}:\quad \mathbf{a}^\top \mathbf{x}\leq c}
	\end{align*}
	and the positive and negative open half spaces
	\begin{align*}
	\mathring{H}_+ = \braces{\mathbf{x}:\quad \mathbf{a}^\top \mathbf{x}> c}\\
	\mathring{H}_- = \braces{\mathbf{x}:\quad \mathbf{a}^\top \mathbf{x}< c}
	\end{align*}
\end{definition}

Half spaces are convex sets and the union of $H_+$ and $H_-$ is the whole space.


\begin{definition}
	A set which can be expressed as the intersection of a finite number of closed half spaces is said to be a convex polytope.
\end{definition}

Convex polytopes are the sets obtained as the family of solutions to a set of linear inequalities of the form,
\begin{align*}
	\mathbf{a}_1^\top \mathbf{x}&\leq b_1 \\
	\mathbf{a}_2^\top \mathbf{x}&\leq b_2 \\
	&\vdots \\
	\mathbf{a}_m^\top \mathbf{x}&\leq b_m,
\end{align*}
since each individual inequality defines a half space and the solution family is the intersection of these half space. 

\begin{definition}
A nonempty bounded polytope is called a \textbf{polyhedron}.
\end{definition}

\begin{theorem}
Let $C$ be convex set and let $\mathbf{y}$ be a point exterior to the closure of $C$. Then there is a vector $\mathbf{a}$ such that $\mathbf{a}^\top \mathbf{y}<\inf_{\mathbf{x}\in C} \mathbf{a}^\top \mathbf{x}$.
\end{theorem}

\begin{theorem}
Let $C$ be a convex set and let $\mathbf{y}$ be a boundary point of $C$. Then there is a hyperplane containing $\mathbf{y}$ and containing $C$ in one of its closed half spaces. 
\end{theorem}

\begin{definition}
A hyperplane containing a convex set $C$ in one of its closed half spaces and containing a boundary point of $C$ is said to be a supporting hyperplane of $C$.
\end{definition}

\begin{theorem}
Let $B$ and $C$ be convex sets with no common relative interior points. Then there is a hyperplane separating $B$ and $D$. In particular, there is a nonzero vector $\mathbf{a}$ such that $\sup_{\mathbf{b}\in B}\mathbf{a}^\top \mathbf{b}\leq \inf_{\mathbf{c}\in C} \mathbf{a}^\top \mathbf{c}$. 
\end{theorem}

\begin{theorem}
	Let $C$ be a convex set, $H$ a supporting hyperplane of $C$, and $T$ the intersection of $H$ and $C$. Every extreme point of $T$ is an extreme point of $C$.
\end{theorem}

\begin{theorem}
A closed bounded convex set in $\Real^n$ is equal to the closed convex hull of its extreme point.
\end{theorem}

\begin{theorem}
A convex polyhedron can be described either as a bounded intersection of a finite number of closed half spaces, or a s the convex hull of a finite number of points. 
\end{theorem}