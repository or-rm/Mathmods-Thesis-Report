\chapter{Computation of an Optimal Transport}
The approximation of an optimal transport is a challenging problem, computationally speaking. We have found a rich literature on it, and many recent advances in this topic have arisen in the very last years. 
%We can attack this problem by the approaches of \textbf{First  Discretize then Optimize} or \textbf{First Optimize then Discretize}
\section{Linear Programming Formulation.}
Let $X$ and $Y$ be two finite sets having $n$ and $m$ elements respectively. Let $\mu$ a probability measure defined over $X$,
\begin{equation}
	\mu=\sum_{i=1}^{n} a_i \delta_{x_i},
\end{equation} 
where $X=\braces{x_1, x_2, \dots, x_n}$ and $0\leq\mathbf{a}=\braces{a_1, a_2, \dots, a_n}$, and $\sum_{i=1}^{n}a_i=1$.

Let $\nu$ a probability measure defined over $Y$,
\begin{equation}
	\nu=\sum_{i=1}^{m} b_i \delta_{y_i},
\end{equation}
where $Y=\braces{y_1, y_2, \dots, y_m}$ and $0\leq\mathbf{b}=\braces{b_1, b_2, \dots, b_m}$, and $\sum_{i=1}^{m}b_i=1$. 
Let $(\pmb{\gamma})_{i,j}=\gamma_{i,j}$ be a joint probability distribution with marginals given by $\mu$ and $\nu$. That is,
\begin{align}
	P(x_i|Y)=\sum_{j=1}^{m}\gamma_{i,j}=a_i\\ P(y_j|X)=\sum_{i=1}^{n}\gamma_{i,j}=b_j
\end{align}
Let $c:X\times Y \rightarrow \Real$ a cost function. Since we $X$ and $Y$ are finite dimensional, we find convenient to use the following notation,
\begin{equation}
	(\mathbf C)_{i,j}=c_{i,j}=c(x_i, y_j)
\end{equation}
Given the matrix nature of the formulation for discrete measures we can compute the total cost by the Frobenius inner product of the two matrices, since $\mathbf{C}$ and $\pmb{\gamma}$ have the same dimensions.
\begin{equation}
	\anglesbox{\mathbf{C}, \pmb{\gamma}}=\sum_{\substack{1\leq i \leq n \\ 1\leq j\leq m}}c_{i,j}\gamma_{i,j}=\tr\parentheses{\mathbf{C}^\top\pmb{\gamma}}
\end{equation}
\subsection{Simplex Method Algorithm and Duality.}
We can solve the problem using the simplex method.
\begin{equation}
	\argmin{\gamma\in \Pi(\mu, \nu)}{\anglesbox{\mathbf{C}, \pmb{\gamma}}}
\end{equation}
\subsection{The simplex method is Not polynomial time.}
Consider the following example
\begin{align}
	\max \sum_{j=1}^{n} 10^{n-j}x_j\\
	\subject 2\sum_{j=1}^{i-1}10^{i-j}x_j+x_i \leq 
\end{align}

\subsection{Sinkhorn-Knopp Algorithm.}
Consider the problem adding a regularization.
\begin{equation}
	\argmin{\gamma\in \Pi(\mu, \nu)}{\anglesbox{\mathbf{C},\pmb{\gamma}}+H(\pmb{\gamma})}
\end{equation}
Where $H(\gamma)$ is the Shannon's Entropy defined for a matrix,
\begin{equation}
	H(\pmb{\gamma})=-\sum_{\substack{1\leq i \leq n\\ 1\leq j \leq m}}\gamma_{i,j}\log(\gamma_{i,j})
\end{equation} 
\section{Continuous Formulation.}
\subsection{Beckman Problem and Optimal Transport.}
\subsection{Proximal Splitting Algorithms.}

